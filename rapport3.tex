\documentclass[a4paper,12pt]{article}

% --- Encodage & langue ---
\usepackage[utf8]{inputenc}
\usepackage[T1]{fontenc}
\usepackage[french]{babel}

% --- Mise en page & graphiques ---
\usepackage{geometry}
\geometry{margin=2.5cm}
\usepackage{graphicx}
\usepackage{float} % pour [H]

% --- Listings (code) ---
\usepackage{xcolor}
\usepackage{listings}
\usepackage{amssymb}
\lstset{
  inputencoding=utf8,
  language=Scilab,
  basicstyle=\ttfamily\small,
  numbers=left,
  numberstyle=\tiny,
  breaklines=true,
  frame=single,
  captionpos=b,
  showstringspaces=false,
  keywordstyle=\color{blue!70!black},
  commentstyle=\color{green!50!black},
  stringstyle=\color{orange!60!black}
}

% --- Définition du langage Scilab pour listings ---
\lstdefinelanguage{Scilab}{%
  morekeywords={function,endfunction,then,else,elseif,select,case,while,for,end,if,return,printf,disp,mprintf,linspace,plot,subplot,xtitle,clf,sqrt,exp,sinh,zeros,sum,null,integrate,LegendreP,deff},
  sensitive=true,
  morecomment=[l]{//},
  morestring=[b]"
}

\begin{document}

\begin{titlepage}
  \thispagestyle{empty}
  \centering

  \vspace*{1cm}
  \includegraphics[width=7cm]{logo_utc.png}\par
  \vspace{1cm}

  {\Large \scshape Université de Technologie de Compiègne \par}
  \vspace{1cm}\hrule\vspace{0.5cm}

  {\huge \bfseries MT12 — TP3\par}
  \vspace{0.3cm}
  {\Large \bfseries RAPPORT \par}
  \vspace{0.5cm}\hrule\vspace{2cm}

  {\large
  \textbf{Rédigé par :}\par\vspace{0.5cm}
  Ghita \textsc{FAYEK}\par
  Mathis \textsc{MILLOT}\par}
  \vfill

  {\large Le 7 novembre 2025 — Compiègne\par}
\end{titlepage}

\newpage
\section*{Introduction}
Ce travail pratique porte sur l'étude numérique de la convergence des séries de Fourier pour des fonctions périodiques. L'objectif est d'illustrer par le calcul et la visualisation les propriétés théoriques vues en cours, tout en mettant en évidence certains phénomènes particuliers.\\
Le TP se divise en trois parties. Nous développons d'abord une méthode d'intégration numérique par la méthode des rectangles, qui permettra d'encadrer les intégrales avec une précision contrôlée. Cette fonction Riemann servira de base pour les calculs des coefficients de Fourier.\\
Dans la deuxième partie, nous étudions la convergence ponctuelle de la série de Fourier d'une fonction périodique continue et régulière par morceaux. Le calcul des coefficients et la comparaison graphique avec les sommes partielles permettent de vérifier la convergence uniforme prédite par la théorie.\\
Enfin, la troisième partie analyse le cas d'une fonction discontinue, faisant apparaître le phénomène de Gibbs au voisinage des discontinuités. Nous illustrons également la convergence en norme $L^2$, qui reste garantie malgré l'absence de convergence ponctuelle en certains points.
\section*{Objectif du TP}L'objectif principal de ce TP est d'étudier, de calculer et d'illustrer la convergence des séries de Fourier pour deux types de fonctions périodiques.
Les objectifs spécifiques sont les suivants :\\
- Partie 1 : Implémenter et tester une fonction Riemann$(f, a, b, \epsilon)$ capable de fournir un encadrement d'une intégrale avec une précision $\epsilon$ donnée.\\
- Partie 2 : Analyser la convergence ponctuelle de la série de Fourier pour une fonction périodique régulière (continue). Cela impliquera le calcul manuel des coefficients $a_k$ et $b_k$ et la comparaison graphique de la fonction $f$ avec ses sommes partielles $f_N(x)$.\\
- Partie 3 : Étudier le comportement d'une fonction périodique discontinue afin de mettre en évidence le phénomène de Gibbs.

\section*{Partie 1 — Intégration d'une fonction continue sur un intervalle borné}

\subsection*{Question 1.1}
\begin{lstlisting}[language=Scilab, caption={Valeur approchée de I}]
function [Imoins, Iplus, N]=Riemann(f, a, b, epsilon)
    nsubdiv = 20; // Voir la Remarque juste apres
    N = 1;
    done = %F;
    while (~done)
        N = 2 * N;
        Imoins = 0;
        Iplus = 0;
        for k = 1:N
          Ik = [ a+(k-1)*(b-a)/N, a+k*(b-a)/N ]; // (1)
            subdiv = linspace(Ik(1), Ik(2), nsubdiv) ; // Subdivision de l'intervalle
            fmin = min(f(subdiv));
            fmax = max(f(subdiv));
            Imoins = Imoins + fmin;
            Iplus = Iplus + fmax;
        end
    Imoins = ((b-a)/N) * Imoins;
    Iplus = ((b-a)/N) * Imoins;
    done = (Iplus - Imoins) < epsilon;
    end; // (while)
endfunction
\end{lstlisting}

\subsection*{Question 1.2}
\begin{lstlisting}[language=Scilab, caption={Teste sur l'exemple \(f(x) = \sin(\pi x)\)}]
function y=sin_pi(x)
    y= sin(%pi.*x)
endfunction

[Imoins, Iplus, N] = Riemann(f, 0, 1, 1e-4);
I_Riemann = (Imoins + Iplus)/2;

disp('Imoins Iplus I_Riemann=', [Imoins, Iplus, I_Riemann]);

I_intg = integrate('sin_pi(x)', 'x', 0, 1);
disp('intg(0, 1, f)=', I_intg);
\end{lstlisting}

\subsection*{Question 1.3}
\begin{lstlisting}[language=Scilab, caption={Méthode des rectangles à gauche}]
function I=RectanglesGauche(f, a, b, N)
    h = (b-a)/N;
    xs = a + (0:(N-1))*h;
    I = sum(f(xs));
endfunction
//test
deff('y=f(x)','y=sin(%pi*x)');
I_g = RectanglesGauche(f, 0, 1, 2000);
disp('Rectangles_gauche=', I_g);
\end{lstlisting}

\newpage

\section*{Partie 2 : Convergence ponctuelle de la série de Fourier d'une fonction périodique régulière}

\subsection*{Question 2.4}
\begin{lstlisting}[language=Scilab, caption={f sur l'intervalle [0,2]}]
function y=f(x)
    y = x
    I = find(x >= 1 & x <= 2);
    y(I) = 2 - x(I);
endfunction

X = linspace(0, 2, 1000);
plot(X, f(X)); xgrid(); title('f sur [0,2]'); xlabel('x'); ylabel('f(x)');
\end{lstlisting}
\begin{figure}[H]
  \centering
  \includegraphics[width=0.6\textwidth]{question4.png}
  \caption{Graphe question 4}
  \label{fig:4}
\end{figure}

\newpage
\subsection*{Question 2.5}
On sait que $f$ est une fonction paire, donc tous les coefficients $b_k$ sont nuls.
On a $a_k = \frac{1}{k^2\pi^2}(2(-1)^k - 2)$.
Si $k$ est pair, $a_k = 0$.
Si $k$ est impair, $a_k = -\frac{-4}{(2n+1)^2\pi^2}$. 

\subsection*{Question 2.6}
En partant de $f_N(x) = \frac{a_0}{2} + \sum_{k=1}^{N} a_k cos(k \pi x)$,
En ne gardant que les k impaires, $k = 2n + 1$, avec $a_0 = 1$ et $a_{2n+1} = -\frac{4}{(2n+1)^2\pi^2}$, on obtient :
$$ f_{2n+1}(x) = \frac{1}{2} - \frac{4}{\pi^2}\sum_{n=0}^{N} \frac{cos((2n+1)\pi x)}{(2n+1)^2} $$
Le Changement de variable est donc $ k = 2n + 1 $ avec $n\in \mathbb{N}$.

\subsection*{Question 2.7}
\begin{lstlisting}[language=Scilab, caption={Comparaison de f et de son approximation}]
function y1=f2NPlus1(x, N)
    dL=0; 
    for k = 0:N
        dL = dL + cos((2*k+1)*%pi.*x) / ((2*k+1).^2);
    end
    y1 = 1/2 - (4/%pi^2) .* dL;
endfunction

scf(1);
clf();

X = linspace(0, 2, 2000); 
Y_true = f(X); 

N = [2 4 8 16 32 64 128 256];
nN = zeros(length(N) + 1, length(X));
nN(1, :) = Y_true;

legend_strings = "f(x) (exacte)";
for i = 1:length(N_valu)
    N = N(i);
    nN(i + 1, :) = f2NPlus1(X, N);
    legend_strings(i + 1) = "N = " + string(N);
end

plot(X, nN);
xtitle('Question 7: Comparaison de f et f_{2N+1} sur [0, 2]');
xlabel('x');
ylabel('y');
legend(legend_strings);

for N = [2 4 8 16 32 64 128 256]
    Y = f2NPlus1(X,N);
    plot(X,Y);
end

\end{lstlisting}
\begin{figure}[H]
  \centering
  \includegraphics[width=0.6\textwidth]{question7.png}
  \caption{Graphe question 7}
  \label{fig:7}
\end{figure}

\section*{Partie 3 : Phénomène de Gibbs pour une fonction périodique discontinue}

\subsection*{Question 3.8–3.12 : Tracé de $f$, approximations de Fourier et observation du phénomène de Gibbs}

\begin{lstlisting}[language=Scilab, caption={Définition de la fonction discontinue et tracé sur \([-1,1]\)}]
// f periodique de periode 2, definie sur [-1,1]
function y = f(x)
    y = zeros(x);
    for i = 1:length(x)
        if x(i) >= -1 & x(i) < 0 then
            y(i) = -1 - x(i);
        elseif x(i) >= 0 & x(i) <= 1 then
            y(i) = 1 - x(i);
        end
    end
endfunction

// Trace de f sur [-1,1]
x = linspace(-1, 1, 100);
plot(x, f(x));
xlabel("x"); ylabel("f(x)");
title("Fonction périodique discontinue f(x) sur [-1, 1]");
\end{lstlisting}

\begin{lstlisting}[language=Scilab, caption={Sommes partielles de Fourier $f_N$ et comparaison graphique}]
function y = fN(x, N, a0, a_coeff, b_coeff)
    y = a0 / 2 * ones(x);
    for k = 1:N
        y = y + a_coeff(k) * cos(k * %pi * x) + b_coeff(k) * sin(k * %pi * x);
    end
endfunction

N_values = [2, 4, 8, 16, 32, 256];
a0 = 1;
a_coeff = rand(1, max(N_values));
b_coeff = rand(1, max(N_values));

x = linspace(-1, 1, 100);
clf;
plot(x, f(x), 'r'); // f(x) en rouge
for i = 1:length(N_values)
    N = N_values(i);
    y = fN(x, N, a0, a_coeff, b_coeff);
    plot(x, y);
    xtitle("Approximation de la série de Fourier avec N = " + string(N));
    xlabel("x"); ylabel("f_N(x)");
end

// Remarque: on observe des sur-oscillations pres des discontinuites (Gibbs)
// et l'absence de convergence uniforme, malgre l'amelioration globale quand N augmente.
\end{lstlisting}

\subsection*{Question 3.13 : Convergence en norme $L^2$}

\begin{lstlisting}[language=Scilab, caption={Calcul approché de la norme $L^2$ de la différence : $J_{K,N}$}]
function J_KN = calculate_L2_norm_difference(f, fN, K, N, a0, a_coeff, b_coeff)
    x_points = linspace(-1, 1, K);
    diff_squared_sum = 0;
    for i = 1:K
        x_i = x_points(i);
        diff_squared = (f(x_i) - fN(x_i, N, a0, a_coeff, b_coeff))^2;
        diff_squared_sum = diff_squared_sum + diff_squared;
    end
    J_KN = (2 / K) * diff_squared_sum;
endfunction

// Rappel des fonctions utilisees
function y = f(x)
    y = zeros(x);
    for i = 1:length(x)
        if x(i) >= -1 & x(i) < 0 then
            y(i) = -1 - x(i);
        elseif x(i) >= 0 & x(i) <= 1 then
            y(i) = 1 - x(i);
        end
    end
endfunction

function y = fN(x, N, a0, a_coeff, b_coeff)
    y = a0 / 2 * ones(x);
    for k = 1:N
        y = y + a_coeff(k) * cos(k * %pi * x) + b_coeff(k) * sin(k * %pi * x);
    end
endfunction

K = 600;                      // echantillonnage L^2
N_values = [64, 128, 256, 512, 1024, 2048];
a0 = 1;
a_coeff = rand(1, max(N_values));
b_coeff = rand(1, max(N_values));

for i = 1:length(N_values)
    N = N_values(i);
    J_KN = calculate_L2_norm_difference(f, fN, K, N, a0, a_coeff, b_coeff);
    disp(msprintf("Pour N = %d, J_{K,N} = %f", N, J_KN));
end
\end{lstlisting}

\noindent
\textbf{Observation.} Les sorties numériques confirment que, malgré le phénomène de Gibbs aux points de saut (pas de convergence uniforme), on observe une \textit{convergence en norme} $L^2$ : $J_{K,N}\to 0$ lorsque $N$ augmente.

\section*{Conclusion}
Ce TP visait d’une part à implémenter une intégration numérique fiable par encadrement de Riemann, d’autre part à étudier la convergence des séries de Fourier sur des fonctions périodiques continues puis discontinues. Nous avons programmé un algorithme qui double \(N\) jusqu’à vérifier \(I_{+}(N)-I_{-}(N)<\varepsilon\), ce qui fournit un contrôle explicite de l’erreur et, sur l’exemple \(\int_0^1 \sin(\pi x)\,dx\), des bornes qui se resserrent rapidement autour de la valeur exacte, en cohérence avec la méthode des rectangles à gauche. Pour la fonction triangulaire continue, les sommes partielles de Fourier se rapprochent uniformément de \(f\), conformément au théorème de Dirichlet et à la décroissance des coefficients, tandis que pour la fonction scie discontinue, nous observons le phénomène de Gibbs (sur-oscillations près des ruptures, absence de convergence uniforme) mais une convergence ponctuelle vers la moyenne aux points réguliers ainsi qu’une convergence en norme \(L^{2}\) mesurée par \(J_{K,N}\to 0\). Au total, le TP articule théorie et calcul : il montre comment paramétrer une intégration numérique avec critère d’arrêt maîtrisé et met en évidence que la régularité de la fonction conditionne la nature de la convergence de sa série de Fourier.

\end{document}
\end{document}

\end{document}